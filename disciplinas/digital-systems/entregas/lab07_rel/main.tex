\documentclass[12pt, a4paper]{article}
\usepackage[utf8]{inputenc}
\usepackage[brazilian]{babel}
\usepackage{hyperref}
\usepackage{listings}
\usepackage{tikz}
\usepackage{tikz-timing}
\usepackage{xparse}

\usetikztiminglibrary[falling arrows]{clockarrows}

\NewDocumentCommand{\busref}{som}{\texttt{%
#3%
\IfValueTF{#2}{[#2]}{}%
\IfBooleanTF{#1}{\#}{}%
}}

\lstset{
	frame=single,
	aboveskip=3mm,
	belowskip=3mm,
	showstringspaces=false,
	columns=flexible,
	basicstyle={\small\ttfamily},
	numbers=none,
	keywordstyle=\color{blue},
	breaklines=true,
	breakatwhitespace=true,
	tabsize=3
}

\author{Fernando Henrique Ratusznei Caetano}
\title{Relatório do Laboratório 7}

\begin{document}
\maketitle
\tableofcontents

\section{Objetivo}
\par A proposta do laboratório 7 foi o estudo de um circuito empregando uma memória ROM 256x8,
isto é, uma memória armazenando 256 bytes.
Dada uma string binária de 7 bits, o objetivo era projetar um circuito capaz
a detectar nos dados armazenados na memória.

\par O principal desafio desse laboratório foi a sincronização dos sinais de CLOCK para que
o sistema exibisse o funcionamento correto.

\newpage
\section{Diagrama em blocos}

\par O projeto foi todo desenvolvido em VHDL. 
A seguir é apresentado o diagrama simplificado com os blocos funcionais do circuito.
Os conversores HEX-7 segmentos e o divisor de clock foram omitidos do diagrama simplificado.

% Important: If latex complains about unicode characters,
% please use "\usepackage[utf8x]{inputenc}" in your preamble
% You can change the size of the picture by putting it into the construct:
% 1) \resizebox{10cm}{!}{"below picture"} to scale horizontally to 10 cm
% 2) \resizebox{!}{15cm}{"below picture"} to scale vertically to 15 cm
% 3) \resizebox{10cm}{15cm}{"below picture"} a combination of above two
% It is not recomended to use the scale option of the tikzpicture environment.
\resizebox{\textwidth}{!}{
\begin{tikzpicture}[x=1pt,y=-1pt,line cap=rect]
	\def\logisimfontA#1{\fontfamily{cmr}{#1}} % Replaced by logisim, original font was "SansSerif"
	\def\logisimfontB#1{\fontfamily{cmtt}{#1}} % Replaced by logisim, original font was "Monospaced"
	\definecolor{custcol_0_64_0}{RGB}{0, 100, 0}
	\definecolor{custcol_0_0_0}{RGB}{0, 0, 0}
	\definecolor{custcol_c0_0_0}{RGB}{192, 0, 0}
	\definecolor{custcol_40_40_40}{RGB}{64, 64, 64}
	\definecolor{custcol_ff_ff_ff}{RGB}{255, 255, 255}
	\definecolor{custcol_ff_0_0}{RGB}{255, 0, 0}
	\definecolor{custcol_28_28_ff}{RGB}{40, 40, 255}
	\draw [line width=3.0pt, custcol_28_28_ff ]  (440.0,175.0) -- (470.0,175.0) ;
	\draw [line width=3.0pt, custcol_28_28_ff ]  (250.0,325.0) -- (300.0,325.0) -- (300.0,175.0) -- (310.0,175.0) ;
	\draw [line width=4.0pt, custcol_0_0_0 ]  (150.0,165.0) -- (160.0,165.0) -- (180.0,165.0) ;
	\draw [line width=4.0pt, custcol_0_0_0 ]  (180.0,95.0) -- (190.0,95.0) -- (190.0,65.0) ;
	\draw [line width=4.0pt, custcol_0_0_0 ]  (580.0,65.0) -- (580.0,75.0) -- (610.0,75.0) -- (610.0,175.0) -- (600.0,175.0) ;
	\draw [line width=4.0pt, custcol_0_0_0 ]  (180.0,85.0) -- (230.0,85.0) -- (230.0,65.0) ;
	\draw [line width=4.0pt, custcol_0_0_0 ]  (360.0,65.0) -- (360.0,85.0) -- (310.0,85.0) ;
	\draw [line width=4.0pt, custcol_0_0_0 ]  (290.0,105.0) -- (290.0,165.0) -- (310.0,165.0) ;
	\draw [line width=3.0pt, custcol_28_28_ff ]  (100.0,305.0) -- (100.0,345.0) ;
	\draw [line width=3.0pt, custcol_28_28_ff ]  (310.0,185.0) -- (290.0,185.0) -- (290.0,225.0) -- (450.0,225.0) -- (450.0,185.0) -- (470.0,185.0) ;
	\draw [line width=3.0pt, custcol_28_28_ff ]  (250.0,305.0) -- (460.0,305.0) -- (460.0,195.0) -- (470.0,195.0) ;
	\draw [line width=3.0pt, custcol_28_28_ff ]  (290.0,225.0) -- (290.0,335.0) -- (250.0,335.0) ;
	\draw [line width=4.0pt, custcol_0_0_0 ]  (440.0,165.0) -- (450.0,165.0) -- (450.0,35.0) -- (460.0,35.0) ;
	\draw [line width=4.0pt, custcol_0_0_0 ]  (280.0,165.0) -- (290.0,165.0) ;
	\draw [line width=4.0pt, custcol_0_0_0 ]  (310.0,95.0) -- (320.0,95.0) -- (320.0,65.0) ;
	\draw [line width=3.0pt, custcol_28_28_ff ]  (250.0,315.0) -- (270.0,315.0) -- (270.0,205.0) -- (170.0,205.0) -- (170.0,175.0) -- (180.0,175.0) ;
	\draw [line width=3.0pt, custcol_28_28_ff ]  (250.0,295.0) -- (260.0,295.0) -- (260.0,215.0) -- (40.0,215.0) -- (40.0,165.0) -- (50.0,165.0) ;
	\draw [line width=3.0pt, custcol_28_28_ff ]  (30.0,295.0) -- (40.0,295.0) -- (40.0,335.0) -- (50.0,335.0) ;
	\draw [line width=3.0pt, custcol_28_28_ff ]  (80.0,345.0) -- (90.0,345.0) -- (90.0,305.0) -- (100.0,305.0) -- (100.0,295.0) -- (70.0,295.0) ;
	\draw [line width=4.0pt, custcol_0_0_0 ]  (160.0,105.0) -- (160.0,165.0) ;
	\draw [line width=3.0pt, custcol_28_28_ff ]  (600.0,185.0) -- (610.0,185.0) -- (610.0,375.0) -- (40.0,375.0) -- (40.0,355.0) -- (50.0,355.0) ;
	\fill [line width=3.0pt, custcol_28_28_ff]  (290.0,225.0) ellipse (5.0 and 5.0 );
	\fill [line width=3.0pt, custcol_0_0_0]  (160.0,165.0) ellipse (6.0 and 6.0 );
	\fill [line width=3.0pt, custcol_28_28_ff]  (100.0,295.0) ellipse (5.0 and 5.0 );
	\fill [line width=3.0pt, custcol_0_0_0]  (290.0,165.0) ellipse (6.0 and 6.0 );
	\fill [line width=3.0pt, custcol_28_28_ff]  (100.0,305.0) ellipse (5.0 and 5.0 );
	\draw [line width=2.0pt, custcol_0_0_0 ]  (470.0,145.0) -- (599.0,145.0) ;
	\draw [line width=2.0pt, custcol_0_0_0 ]  (600.0,145.0) -- (600.0,209.0) ;
	\draw [line width=2.0pt, custcol_0_0_0 ]  (600.0,210.0) -- (471.0,210.0) ;
	\draw [line width=2.0pt, custcol_0_0_0 ]  (470.0,210.0) -- (470.0,146.0) ;
	\logisimfontA{\fontsize{10pt}{10pt}\selectfont\node[inner sep=0, outer sep=0, custcol_40_40_40, anchor=base west] at  (475.0,179.0)  {serial\_in};}
	\logisimfontA{\fontsize{10pt}{10pt}\selectfont\node[inner sep=0, outer sep=0, custcol_40_40_40, anchor=base west] at  (475.0,189.0)  {clock};}
	\logisimfontA{\fontsize{10pt}{10pt}\selectfont\node[inner sep=0, outer sep=0, custcol_40_40_40, anchor=base west] at  (475.0,199.0)  {reset\_not};}
	\logisimfontA{\fontsize{10pt}{10pt}\selectfont\node[inner sep=0, outer sep=0, custcol_40_40_40, anchor=base west] at  (550.0,179.0)  {state\_out};}
	\logisimfontA{\fontsize{10pt}{10pt}\selectfont\node[inner sep=0, outer sep=0, custcol_40_40_40, anchor=base west] at  (540.0,189.0)  {is\_detected};}
	\logisimfontA{\fontsize{12pt}{12pt}\selectfont\node[inner sep=0, outer sep=0, custcol_0_0_0, anchor=base west] at  (482.0,163.0)  {sequence\_detector};}
	\fill [line width=1.0pt, custcol_28_28_ff]  (470.0,175.0) ellipse (2.0 and 2.0 );
	\fill [line width=1.0pt, custcol_28_28_ff]  (470.0,185.0) ellipse (2.0 and 2.0 );
	\fill [line width=1.0pt, custcol_28_28_ff]  (470.0,195.0) ellipse (2.0 and 2.0 );
	\fill [line width=1.0pt, custcol_0_0_0]  (600.0,175.0) ellipse (2.0 and 2.0 );
	\fill [line width=1.0pt, custcol_28_28_ff]  (600.0,185.0) ellipse (2.0 and 2.0 );
	\logisimfontA{\fontsize{12pt}{12pt}\fontseries{bx}\selectfont\node[inner sep=0, outer sep=0, custcol_0_0_0, anchor=base west] at  (9.0,277.0)  {PB0};}
	\draw [line width=2.0pt, custcol_0_0_0 ]  (50.0,330.0) -- (79.0,330.0) ;
	\draw [line width=2.0pt, custcol_0_0_0 ]  (80.0,330.0) -- (80.0,359.0) ;
	\draw [line width=2.0pt, custcol_0_0_0 ]  (80.0,360.0) -- (51.0,360.0) ;
	\draw [line width=2.0pt, custcol_0_0_0 ]  (50.0,360.0) -- (50.0,331.0) ;
	\logisimfontA{\fontsize{12pt}{12pt}\selectfont\node[inner sep=0, outer sep=0, custcol_0_0_0, anchor=base west] at  (60.0,345.0)  {\&};}
	\fill [line width=2.0pt, custcol_28_28_ff]  (80.0,345.0) ellipse (2.0 and 2.0 );
	\fill [line width=2.0pt, custcol_28_28_ff]  (50.0,335.0) ellipse (2.0 and 2.0 );
	\fill [line width=2.0pt, custcol_28_28_ff]  (50.0,355.0) ellipse (2.0 and 2.0 );
	\logisimfontA{\fontsize{12pt}{12pt}\fontseries{bx}\selectfont\node[inner sep=0, outer sep=0, custcol_0_0_0, anchor=base west] at  (38.0,277.0)  {CLK\_DIV};}
	\draw [line width=2.0pt, custcol_0_0_0 ]  (180.0,135.0) -- (279.0,135.0) ;
	\draw [line width=2.0pt, custcol_0_0_0 ]  (280.0,135.0) -- (280.0,189.0) ;
	\draw [line width=2.0pt, custcol_0_0_0 ]  (280.0,190.0) -- (181.0,190.0) ;
	\draw [line width=2.0pt, custcol_0_0_0 ]  (180.0,190.0) -- (180.0,136.0) ;
	\logisimfontA{\fontsize{10pt}{10pt}\selectfont\node[inner sep=0, outer sep=0, custcol_40_40_40, anchor=base west] at  (185.0,169.0)  {address};}
	\logisimfontA{\fontsize{10pt}{10pt}\selectfont\node[inner sep=0, outer sep=0, custcol_40_40_40, anchor=base west] at  (185.0,179.0)  {clock};}
	\logisimfontA{\fontsize{10pt}{10pt}\selectfont\node[inner sep=0, outer sep=0, custcol_40_40_40, anchor=base west] at  (269.0,169.0)  {q};}
	\logisimfontA{\fontsize{12pt}{12pt}\selectfont\node[inner sep=0, outer sep=0, custcol_0_0_0, anchor=base west] at  (197.0,153.0)  {ROM\_256\_8};}
	\fill [line width=1.0pt, custcol_0_0_0]  (180.0,165.0) ellipse (2.0 and 2.0 );
	\fill [line width=1.0pt, custcol_28_28_ff]  (180.0,175.0) ellipse (2.0 and 2.0 );
	\fill [line width=1.0pt, custcol_0_0_0]  (280.0,165.0) ellipse (2.0 and 2.0 );
	\draw [line width=2.0pt, custcol_0_0_0 ]  (50.0,285.0) -- (69.0,285.0) ;
	\draw [line width=2.0pt, custcol_0_0_0 ]  (70.0,285.0) -- (70.0,304.0) ;
	\draw [line width=2.0pt, custcol_0_0_0 ]  (70.0,305.0) -- (51.0,305.0) ;
	\draw [line width=2.0pt, custcol_0_0_0 ]  (50.0,305.0) -- (50.0,286.0) ;
	\draw [line width=2.0pt, custcol_0_64_0 ]  (54.0,295.0) -- (54.0,299.0) -- (60.0,299.0) -- (60.0,291.0) -- (66.0,291.0) -- (66.0,295.0) ;
	\fill [line width=2.0pt, custcol_28_28_ff]  (70.0,295.0) ellipse (2.0 and 2.0 );
	\draw [line width=2.0pt, custcol_0_0_0 ]  (305.0,5.0) -- (344.0,5.0) ;
	\draw [line width=2.0pt, custcol_0_0_0 ]  (305.0,65.0) -- (305.0,6.0) ;
	\fill [line width=1.0pt, custcol_ff_0_0 ]  (313.0,13.0) rectangle (332.0,17.0) ;
	\fill [line width=1.0pt, custcol_ff_0_0 ]  (333.0,15.0) rectangle (337.0,34.0) ;
	\fill [line width=1.0pt, custcol_ff_0_0 ]  (333.0,35.0) rectangle (337.0,54.0) ;
	\fill [line width=1.0pt, custcol_ff_0_0 ]  (313.0,52.0) rectangle (332.0,56.0) ;
	\fill [line width=1.0pt, custcol_ff_0_0 ]  (308.0,35.0) rectangle (312.0,54.0) ;
	\fill [line width=1.0pt, custcol_ff_0_0 ]  (308.0,15.0) rectangle (312.0,34.0) ;
	\fill [line width=1.0pt, custcol_ff_0_0 ]  (313.0,33.0) rectangle (332.0,37.0) ;
	\fill [line width=1.0pt, custcol_ff_0_0]  (340.0,55.0) ellipse (2.5 and 2.5 );
	\fill [line width=1.0pt, custcol_0_0_0]  (320.0,65.0) ellipse (2.0 and 2.0 );
	\fill [line width=1.0pt, custcol_28_28_ff]  (340.0,65.0) ellipse (2.0 and 2.0 );
	\draw [line width=2.0pt, custcol_0_0_0 ]  (100.0,265.0) -- (249.0,265.0) ;
	\draw [line width=2.0pt, custcol_0_0_0 ]  (250.0,265.0) -- (250.0,349.0) ;
	\draw [line width=2.0pt, custcol_0_0_0 ]  (250.0,350.0) -- (101.0,350.0) ;
	\draw [line width=2.0pt, custcol_0_0_0 ]  (100.0,350.0) -- (100.0,266.0) ;
	\logisimfontA{\fontsize{10pt}{10pt}\selectfont\node[inner sep=0, outer sep=0, custcol_40_40_40, anchor=base west] at  (105.0,299.0)  {clk\_geral};}
	\logisimfontA{\fontsize{10pt}{10pt}\selectfont\node[inner sep=0, outer sep=0, custcol_40_40_40, anchor=base west] at  (105.0,309.0)  {enable\_not};}
	\logisimfontA{\fontsize{10pt}{10pt}\selectfont\node[inner sep=0, outer sep=0, custcol_40_40_40, anchor=base west] at  (191.0,299.0)  {clk\_counter};}
	\logisimfontA{\fontsize{10pt}{10pt}\selectfont\node[inner sep=0, outer sep=0, custcol_40_40_40, anchor=base west] at  (170.0,309.0)  {reset\_sequence};}
	\logisimfontA{\fontsize{10pt}{10pt}\selectfont\node[inner sep=0, outer sep=0, custcol_40_40_40, anchor=base west] at  (198.0,319.0)  {clk\_memo};}
	\logisimfontA{\fontsize{10pt}{10pt}\selectfont\node[inner sep=0, outer sep=0, custcol_40_40_40, anchor=base west] at  (200.0,329.0)  {piso\_load};}
	\logisimfontA{\fontsize{10pt}{10pt}\selectfont\node[inner sep=0, outer sep=0, custcol_40_40_40, anchor=base west] at  (201.0,339.0)  {clk\_serial};}
	\logisimfontA{\fontsize{12pt}{12pt}\selectfont\node[inner sep=0, outer sep=0, custcol_0_0_0, anchor=base west] at  (131.0,283.0)  {clock\_controller};}
	\fill [line width=1.0pt, custcol_28_28_ff]  (100.0,295.0) ellipse (2.0 and 2.0 );
	\fill [line width=1.0pt, custcol_28_28_ff]  (100.0,305.0) ellipse (2.0 and 2.0 );
	\fill [line width=1.0pt, custcol_28_28_ff]  (250.0,295.0) ellipse (2.0 and 2.0 );
	\fill [line width=1.0pt, custcol_28_28_ff]  (250.0,305.0) ellipse (2.0 and 2.0 );
	\fill [line width=1.0pt, custcol_28_28_ff]  (250.0,315.0) ellipse (2.0 and 2.0 );
	\fill [line width=1.0pt, custcol_28_28_ff]  (250.0,325.0) ellipse (2.0 and 2.0 );
	\fill [line width=1.0pt, custcol_28_28_ff]  (250.0,335.0) ellipse (2.0 and 2.0 );
	\draw [line width=2.0pt, custcol_0_0_0 ]  (50.0,135.0) -- (149.0,135.0) ;
	\draw [line width=2.0pt, custcol_0_0_0 ]  (150.0,135.0) -- (150.0,179.0) ;
	\draw [line width=2.0pt, custcol_0_0_0 ]  (150.0,180.0) -- (51.0,180.0) ;
	\draw [line width=2.0pt, custcol_0_0_0 ]  (50.0,180.0) -- (50.0,136.0) ;
	\logisimfontA{\fontsize{10pt}{10pt}\selectfont\node[inner sep=0, outer sep=0, custcol_40_40_40, anchor=base west] at  (55.0,169.0)  {clk};}
	\logisimfontA{\fontsize{10pt}{10pt}\selectfont\node[inner sep=0, outer sep=0, custcol_40_40_40, anchor=base west] at  (139.0,169.0)  {q};}
	\logisimfontA{\fontsize{12pt}{12pt}\selectfont\node[inner sep=0, outer sep=0, custcol_0_0_0, anchor=base west] at  (65.0,153.0)  {counter\_256};}
	\fill [line width=1.0pt, custcol_28_28_ff]  (50.0,165.0) ellipse (2.0 and 2.0 );
	\fill [line width=1.0pt, custcol_0_0_0]  (150.0,165.0) ellipse (2.0 and 2.0 );
	\draw [line width=2.0pt, custcol_0_0_0 ]  (255.0,5.0) -- (255.0,64.0) ;
	\draw [line width=2.0pt, custcol_0_0_0 ]  (255.0,65.0) -- (216.0,65.0) ;
	\fill [line width=1.0pt, custcol_ff_0_0 ]  (223.0,13.0) rectangle (242.0,17.0) ;
	\fill [line width=1.0pt, custcol_ff_0_0 ]  (243.0,15.0) rectangle (247.0,34.0) ;
	\fill [line width=1.0pt, custcol_ff_0_0 ]  (243.0,35.0) rectangle (247.0,54.0) ;
	\fill [line width=1.0pt, custcol_ff_0_0 ]  (223.0,52.0) rectangle (242.0,56.0) ;
	\fill [line width=1.0pt, custcol_ff_0_0 ]  (218.0,35.0) rectangle (222.0,54.0) ;
	\fill [line width=1.0pt, custcol_ff_0_0 ]  (218.0,15.0) rectangle (222.0,34.0) ;
	\fill [line width=1.0pt, custcol_ff_0_0 ]  (223.0,33.0) rectangle (242.0,37.0) ;
	\fill [line width=1.0pt, custcol_ff_0_0]  (250.0,55.0) ellipse (2.5 and 2.5 );
	\fill [line width=1.0pt, custcol_0_0_0]  (230.0,65.0) ellipse (2.0 and 2.0 );
	\fill [line width=1.0pt, custcol_28_28_ff]  (250.0,65.0) ellipse (2.0 and 2.0 );
	\draw [line width=3.0pt, custcol_0_0_0 ]  (180.0,85.0) -- (166.0,85.0) ;
	\draw [line width=3.0pt, custcol_0_0_0 ]  (180.0,95.0) -- (166.0,95.0) ;
	\draw [line width=5.0pt, custcol_0_0_0 ]  (166.0,86.0) -- (166.0,99.0) -- (161.0,104.0) ;
	\logisimfontA{\fontsize{7pt}{7pt}\selectfont\node[inner sep=0, outer sep=0, custcol_0_0_0, anchor=base west] at  (169.0,93.0)  {3-0};}
	\logisimfontA{\fontsize{7pt}{7pt}\selectfont\node[inner sep=0, outer sep=0, custcol_0_0_0, anchor=base west] at  (169.0,103.0)  {7-4};}
	\fill [line width=5.0pt, custcol_0_0_0]  (160.0,105.0) ellipse (2.0 and 2.0 );
	\fill [line width=5.0pt, custcol_0_0_0]  (180.0,85.0) ellipse (2.0 and 2.0 );
	\fill [line width=5.0pt, custcol_0_0_0]  (180.0,95.0) ellipse (2.0 and 2.0 );
	\draw [line width=2.0pt, custcol_0_0_0 ]  (12.0,287.0) -- (29.0,287.0) ;
	\draw [line width=2.0pt, custcol_0_0_0 ]  (30.0,287.0) -- (30.0,304.0) ;
	\draw [line width=2.0pt, custcol_0_0_0 ]  (30.0,305.0) -- (13.0,305.0) ;
	\draw [line width=2.0pt, custcol_0_0_0 ]  (12.0,305.0) -- (12.0,288.0) ;
	\fill [line width=2.0pt, custcol_0_64_0]  (20.0,295.0) ellipse (5.5 and 6.5 );
	\logisimfontB{\fontsize{12pt}{12pt}\selectfont\node[inner sep=0, outer sep=0, custcol_ff_ff_ff, anchor=base west] at  (17.0,301.0)  {0};}
	\fill [line width=2.0pt, custcol_28_28_ff]  (30.0,295.0) ellipse (2.0 and 2.0 );
	\draw [line width=2.0pt, custcol_0_0_0 ]  (175.0,5.0) -- (214.0,5.0) ;
	\draw [line width=2.0pt, custcol_0_0_0 ]  (254.0,5.0) -- (215.0,5.0) -- (215.0,64.0) ;
	\draw [line width=2.0pt, custcol_0_0_0 ]  (215.0,6.0) -- (215.0,65.0) -- (176.0,65.0) ;
	\draw [line width=2.0pt, custcol_0_0_0 ]  (175.0,65.0) -- (175.0,6.0) ;
	\fill [line width=1.0pt, custcol_ff_0_0 ]  (183.0,13.0) rectangle (202.0,17.0) ;
	\fill [line width=1.0pt, custcol_ff_0_0 ]  (203.0,15.0) rectangle (207.0,34.0) ;
	\fill [line width=1.0pt, custcol_ff_0_0 ]  (203.0,35.0) rectangle (207.0,54.0) ;
	\fill [line width=1.0pt, custcol_ff_0_0 ]  (183.0,52.0) rectangle (202.0,56.0) ;
	\fill [line width=1.0pt, custcol_ff_0_0 ]  (178.0,35.0) rectangle (182.0,54.0) ;
	\fill [line width=1.0pt, custcol_ff_0_0 ]  (178.0,15.0) rectangle (182.0,34.0) ;
	\fill [line width=1.0pt, custcol_ff_0_0 ]  (183.0,33.0) rectangle (202.0,37.0) ;
	\fill [line width=1.0pt, custcol_ff_0_0]  (210.0,55.0) ellipse (2.5 and 2.5 );
	\fill [line width=1.0pt, custcol_0_0_0]  (190.0,65.0) ellipse (2.0 and 2.0 );
	\fill [line width=1.0pt, custcol_28_28_ff]  (210.0,65.0) ellipse (2.0 and 2.0 );
	\draw [line width=2.0pt, custcol_0_0_0 ]  (345.0,64.0) -- (345.0,5.0) -- (384.0,5.0) ;
	\draw [line width=2.0pt, custcol_0_0_0 ]  (385.0,5.0) -- (385.0,64.0) ;
	\draw [line width=2.0pt, custcol_0_0_0 ]  (385.0,65.0) -- (346.0,65.0) ;
	\draw [line width=2.0pt, custcol_0_0_0 ]  (306.0,65.0) -- (345.0,65.0) -- (345.0,6.0) ;
	\fill [line width=1.0pt, custcol_ff_0_0 ]  (353.0,13.0) rectangle (372.0,17.0) ;
	\fill [line width=1.0pt, custcol_ff_0_0 ]  (373.0,15.0) rectangle (377.0,34.0) ;
	\fill [line width=1.0pt, custcol_ff_0_0 ]  (373.0,35.0) rectangle (377.0,54.0) ;
	\fill [line width=1.0pt, custcol_ff_0_0 ]  (353.0,52.0) rectangle (372.0,56.0) ;
	\fill [line width=1.0pt, custcol_ff_0_0 ]  (348.0,35.0) rectangle (352.0,54.0) ;
	\fill [line width=1.0pt, custcol_ff_0_0 ]  (348.0,15.0) rectangle (352.0,34.0) ;
	\fill [line width=1.0pt, custcol_ff_0_0 ]  (353.0,33.0) rectangle (372.0,37.0) ;
	\fill [line width=1.0pt, custcol_ff_0_0]  (380.0,55.0) ellipse (2.5 and 2.5 );
	\fill [line width=1.0pt, custcol_0_0_0]  (360.0,65.0) ellipse (2.0 and 2.0 );
	\fill [line width=1.0pt, custcol_28_28_ff]  (380.0,65.0) ellipse (2.0 and 2.0 );
	\draw [line width=2.0pt, custcol_0_0_0 ]  (310.0,135.0) -- (439.0,135.0) ;
	\draw [line width=2.0pt, custcol_0_0_0 ]  (440.0,135.0) -- (440.0,199.0) ;
	\draw [line width=2.0pt, custcol_0_0_0 ]  (440.0,200.0) -- (311.0,200.0) ;
	\draw [line width=2.0pt, custcol_0_0_0 ]  (310.0,200.0) -- (310.0,136.0) ;
	\logisimfontA{\fontsize{10pt}{10pt}\selectfont\node[inner sep=0, outer sep=0, custcol_40_40_40, anchor=base west] at  (315.0,169.0)  {parallel\_in};}
	\logisimfontA{\fontsize{10pt}{10pt}\selectfont\node[inner sep=0, outer sep=0, custcol_40_40_40, anchor=base west] at  (315.0,179.0)  {load\_not};}
	\logisimfontA{\fontsize{10pt}{10pt}\selectfont\node[inner sep=0, outer sep=0, custcol_40_40_40, anchor=base west] at  (315.0,189.0)  {clock};}
	\logisimfontA{\fontsize{10pt}{10pt}\selectfont\node[inner sep=0, outer sep=0, custcol_40_40_40, anchor=base west] at  (378.0,169.0)  {parallel\_out};}
	\logisimfontA{\fontsize{10pt}{10pt}\selectfont\node[inner sep=0, outer sep=0, custcol_40_40_40, anchor=base west] at  (388.0,179.0)  {serial\_out};}
	\logisimfontA{\fontsize{12pt}{12pt}\selectfont\node[inner sep=0, outer sep=0, custcol_0_0_0, anchor=base west] at  (363.0,153.0)  {piso};}
	\fill [line width=1.0pt, custcol_0_0_0]  (310.0,165.0) ellipse (2.0 and 2.0 );
	\fill [line width=1.0pt, custcol_28_28_ff]  (310.0,175.0) ellipse (2.0 and 2.0 );
	\fill [line width=1.0pt, custcol_28_28_ff]  (310.0,185.0) ellipse (2.0 and 2.0 );
	\fill [line width=1.0pt, custcol_0_0_0]  (440.0,165.0) ellipse (2.0 and 2.0 );
	\fill [line width=1.0pt, custcol_28_28_ff]  (440.0,175.0) ellipse (2.0 and 2.0 );
	\draw [line width=3.0pt, custcol_0_0_0 ]  (310.0,85.0) -- (296.0,85.0) ;
	\draw [line width=3.0pt, custcol_0_0_0 ]  (310.0,95.0) -- (296.0,95.0) ;
	\draw [line width=5.0pt, custcol_0_0_0 ]  (296.0,86.0) -- (296.0,99.0) -- (291.0,104.0) ;
	\logisimfontA{\fontsize{7pt}{7pt}\selectfont\node[inner sep=0, outer sep=0, custcol_0_0_0, anchor=base west] at  (299.0,93.0)  {3-0};}
	\logisimfontA{\fontsize{7pt}{7pt}\selectfont\node[inner sep=0, outer sep=0, custcol_0_0_0, anchor=base west] at  (299.0,103.0)  {7-4};}
	\fill [line width=5.0pt, custcol_0_0_0]  (290.0,105.0) ellipse (2.0 and 2.0 );
	\fill [line width=5.0pt, custcol_0_0_0]  (310.0,85.0) ellipse (2.0 and 2.0 );
	\fill [line width=5.0pt, custcol_0_0_0]  (310.0,95.0) ellipse (2.0 and 2.0 );
	\draw [line width=2.0pt, custcol_0_0_0 ]  (565.0,5.0) -- (604.0,5.0) ;
	\draw [line width=2.0pt, custcol_0_0_0 ]  (605.0,5.0) -- (605.0,64.0) ;
	\draw [line width=2.0pt, custcol_0_0_0 ]  (605.0,65.0) -- (566.0,65.0) ;
	\draw [line width=2.0pt, custcol_0_0_0 ]  (565.0,65.0) -- (565.0,6.0) ;
	\fill [line width=1.0pt, custcol_ff_0_0 ]  (573.0,13.0) rectangle (592.0,17.0) ;
	\fill [line width=1.0pt, custcol_ff_0_0 ]  (593.0,15.0) rectangle (597.0,34.0) ;
	\fill [line width=1.0pt, custcol_ff_0_0 ]  (593.0,35.0) rectangle (597.0,54.0) ;
	\fill [line width=1.0pt, custcol_ff_0_0 ]  (573.0,52.0) rectangle (592.0,56.0) ;
	\fill [line width=1.0pt, custcol_ff_0_0 ]  (568.0,35.0) rectangle (572.0,54.0) ;
	\fill [line width=1.0pt, custcol_ff_0_0 ]  (568.0,15.0) rectangle (572.0,34.0) ;
	\fill [line width=1.0pt, custcol_ff_0_0 ]  (573.0,33.0) rectangle (592.0,37.0) ;
	\fill [line width=1.0pt, custcol_ff_0_0]  (600.0,55.0) ellipse (2.5 and 2.5 );
	\fill [line width=1.0pt, custcol_0_0_0]  (580.0,65.0) ellipse (2.0 and 2.0 );
	\fill [line width=1.0pt, custcol_28_28_ff]  (600.0,65.0) ellipse (2.0 and 2.0 );
	\fill [line width=1.0pt, custcol_0_0_0]  (460.0,35.0) ellipse (2.0 and 2.0 );
	\fill [line width=1.0pt, custcol_40_40_40 ]  (460.0,20.0) rectangle (540.0,50.0) ;
	\fill [line width=1.0pt, custcol_c0_0_0 ]  (462.0,26.0) rectangle (468.0,44.0) ;
	\fill [line width=1.0pt, custcol_c0_0_0 ]  (472.0,26.0) rectangle (478.0,44.0) ;
	\fill [line width=1.0pt, custcol_c0_0_0 ]  (482.0,26.0) rectangle (488.0,44.0) ;
	\fill [line width=1.0pt, custcol_c0_0_0 ]  (492.0,26.0) rectangle (498.0,44.0) ;
	\fill [line width=1.0pt, custcol_c0_0_0 ]  (502.0,26.0) rectangle (508.0,44.0) ;
	\fill [line width=1.0pt, custcol_c0_0_0 ]  (512.0,26.0) rectangle (518.0,44.0) ;
	\fill [line width=1.0pt, custcol_c0_0_0 ]  (522.0,26.0) rectangle (528.0,44.0) ;
	\fill [line width=1.0pt, custcol_c0_0_0 ]  (532.0,26.0) rectangle (538.0,44.0) ;
\end{tikzpicture}
}


\par O circuito possui apenas duas entradas, o CLK\_DIV (clock dividido) e o PB0 (botão da placa).
O fluxo da informação no circuito começa no counter\_256, um contador de 0 até 255, que realiza
o endereçamento da ROM com os dados fornecidos pelo professor. 
Durante cada passo da contagem é necessário enviar pulsos para todos os componentes na ordem correta.
Primeiro é necessário enviar um sinal de clock para atualizar os dados na saída da ROM.
Após o tempo de propagação da memória o piso recebe um sinal de load e é carregado com os dados endereçados. 
Com a saída do piso estável, são enviados 8 pulsos de clock para o piso e o detector de sequência. 
Caso o detector encontre a sequência o sinal de enable do controlador de clock é desativado e
o circuito suspende a contagem até que o botão da placa seja pressionado. 
Caso contrário o detector de sequências recebe um sinal de reset, o contador é incrementado
e ciclo se repete para outros os bytes armazenados na memória.

\par Todos esses sinais precisam ser acionados exatamente na ordem descrita no paragráfo anterior.
É necessário um sistema para controlar todos de forma correta. Esse controle é realizado pelo
dispositivo clock\_controller.

\newpage
\section{Sincronização dos sinais de CLOCK}

\par O dispositivo clock\_controller é uma máquina de estado responsável por controlar os sinais de
clock, load e reset de todos os componentes. Os pulsos devem ser acionados conforme o diagrama de tempos a seguir.
Os sinais de clock são ativos na borda de descida e os sinais de reset e load são ativos em zero.

\vspace{3ex}

\begin{tikztimingtable}[%
    timing/dslope=0.1,
    timing/.style={x=2ex,y=2ex},
    x=5ex,
    timing/rowdist=3ex,
    timing/name/.style={font=\sffamily\scriptsize}
]
	\busref{clk\_geral}       & H 42{c} \\
	\busref{clk\_counter}     & H 21{L} \\
	\busref{reset\_sequence}  & L 21{H} \\
	\busref{clk\_memo}        & H H L H H 17{H} \\
	\busref{piso\_load}       & H H H L H 17{H} \\
	\busref{clk\_serial}      & H H H H H 8{L H} H \\ 
	\extracode
	\begin{pgfonlayer}{background}
		\begin{scope}[semitransparent ,semithick]
		\vertlines[darkgray,dotted]{0,0.4,...,8.5}
		\end{scope}
	\end{pgfonlayer}
\end{tikztimingtable}

\vspace{3ex}

\par O controle de clock foi implementado em VHDL utilizando o bloco process apresentado a seguir. 
O bloco foi implementado com um contador e a com lógica combinacional para ativar as saídas na ordem correta.

\begin{lstlisting}[language=VHDL]
clock_controller: process(clk_geral) is
  constant N: integer := 25;
  variable counter: integer := 0;
begin
  if (falling_edge(clk_geral)) then
    if enable = '0' then
      if counter = N then
        counter := 0;
      else
        counter := counter + 1;
      end if;
    end if;

    if counter = 0 then
      clk_counter <= '1';
      reset_sequence <= '0';
    else
      clk_counter <= '0';
      reset_sequence <= '1';
    end if;

    if counter = 1 then
      clk_memo <= '0';
    else
      clk_memo <= '1';
    end if;

	-- Esse sinal foi atrasado em alguns ciclos
	-- para garantir que a saida da memoria ja esta estavel
    if counter = 5 then
      piso_load <= '0';
    else
      piso_load <= '1';
    end if;

    case counter is
      when 7|9|11|13|15|17|19|21 => clk_serial <= '0';
      when others => clk_serial <= '1';
    end case;
  end if;
end process clock_controller;
\end{lstlisting}

\newpage
\section{Conclusão e comentários adicionais}

\par Nesse laboratório foram praticados os conceitos de memórias digitais utilizando
a memória do biblioteca IP (\textit{Intellectual property}) do Quartus.
Os dados para serem armazenados foram carregados por meio de um arquivo .hex,
um formato utilizado para descrever dados binários em texto ASCII.
Esse tipo de arquivo é padrão e pode ser utilizado com outros dispositivos.

\par O código para o piso, os conversores HEX-7 segmentos, o contador e o detector de
sequência foram reaproveitados de práticas anteriores. A modularidade com que foram
desenvolvidos facilitou muito reuso desses componentes.

\par Em aula teória foram aprendidas técnicas para sincronização de clock de vários
componentes, como foi necessário nessa prática. Uma das técnicas expostas foi a da
utilização de um contador em anel para acionar o clock de vários componentes em sequência.
Inicialmente esse método foi considerado, porém foi optado por utilizar um contador 
crescente utilizando estados da contagem na entrada de um circuito sequêncial,
cujo as saídas acionam os clocks apenas durante certos momentos da contagem.
Foi decidido assim pois o código do contador, que já havia sido desenvolvido em
práticas anteriores, foi facilmente adapatado.

\end{document}
